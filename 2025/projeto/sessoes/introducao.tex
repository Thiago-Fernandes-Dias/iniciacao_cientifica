\section{Introdução}\label{sec:introduacao}

Nos últimos anos, a popularização de serviços oferecidos única e exclusivamente pela da Internet tem crescido de forma extraordinária. Muitas pessoas realizam pagamentos em sites de lojas diretamente de suas casas e gerenciam suas contas bancárias sem a necessidade de irem até os bancos \cite{seguranca_de_dados}. Dessa forma, surge a necessidade sistemas de autenticação e identificação cada vez mais sofisticados.

Entretanto, na perspectiva de \citeonline{Ryu2023Design}, a maioria dos sistemas de autenticação comumente usados são baseados em senhas estáticas, ou uma combinação entre uma senha e um cartão inteligente ou \textit{tokens}.~\citeonline{Peacock2004} destacam que essa abordagem possui uma série de limitações. Por exemplo, senhas compostas por palavras comuns, sequências numéricas ou alfanuméricas simples que sejam passíveis de memorização são consideradas ``fracas'' no que se refere ao nível de segurança que elas oferecem, pois podem ser descobertas por ataques do tipo ``força bruta''.

~\citeonline{Peacock2004} ainda afirmam que é recomendado que os usuários criem senhas maiores e compostas por uma combinação de números, letras e caracteres especiais para serem menos vulneráveis a ataques. Em adição a isso, as senhas deveriam ser diferentes para cada conta nos sites em que acessam. Todavia, muitas vezes os usuários ainda optam pela utilização de uma mesma senha para todas as suas contas, o que gera um risco de segurança: se apenas um dos sistemas for invadido e tiver os seus dados roubados, o acesso aos dados do indivíduo em todos os outros serviços estará comprometido.

Segundo \citeonline{biometrics_a_tool_for_information_security}, muitas dessas limitações associadas ao uso de senhas podem ser contornadas pela incorporação de métodos de autenticação melhores. Nesse sentido, a biometria, como uma forma de estabelecer a identidade por meio de características físicas ou comportamentais dos indivíduos, surge como uma alternativa. Há diversas modalidades biométricas~\cite{Jain2016_50Years} que podem ser usadas tanto para identificação quanto para a verificação de usuários.

Neste projeto, será dado o enfoque na vertente física, em particular, na biometria através de impressões digitais. Essa modalidade biométrica se baseia nos padrões de pequenas elevações e depressões na pele da pontas dos dedos~\cite{HandbookOfBiometrics}. Os usuários são então reconhecidos com base nas diferenças nesses padrões.

Conforme definido em~\cite{Jain2004_IntroToBiometrics}, sistemas biométricos são sistemas de reconhecimento de padrões que extraem características de dados biométricos e então comparam as características extraídas com uma referência biométrica em um banco de dados. Nesse sentido,~\citeonline{Maltoni20221} explica que, para a autenticação através de digitais, a representação das características extraídas das imagens é um problema fundamentação da verificação de digitais. 

Como não há a garantia de que as imagens da digital de um mesmo dedo nunca serão exatamente iguais em termos de intensidade de pixel, orientação e formato, há a necessidade de se extrair características a partir das quais seja possível diferenciar diferentes digitais com uma boa acurácia e que sejam invariantes para uma dada digital. Somente assim o uso de digitais em sistemas de autenticação se tornariam viáveis.

O objetivo deste trabalho é avaliar diferentes algoritmos e técnicas de extração de características de imagens de digitais e formas de representação. Sua avaliação será feita com base na sua performance em um cenário de verificação de usuários.

As demais seções do projeto estão organizadas da seguinte forma: na Seção \ref{sec:fundamentacao_teorica}, são introduzidos conceitos sobre digitais e as diferentes representações que podem ser extraídas delas; na Seção \ref{sec:objetivos}, são apresentados os objetivos; na Seção \ref{sec:metodologia}, é descrita a metodologia; na Seção \ref{sec:viabilidade}, a viabilidade do projeto é discutida; e, na Seção \ref{sec:cronograma}, é apresentado o cronograma deste projeto.
