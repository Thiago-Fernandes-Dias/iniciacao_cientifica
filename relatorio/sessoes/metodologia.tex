\section{Metodologia}\label{sec:metodologia}

Este projeto irá comparar técnicas para ajuste de hiperparâmetros de algoritmos de classificação em dinâmica da digitação. Para isso, serão utilizados conjuntos de dados disponíveis publicamente, conforme descrito na Seção~\ref{subsec:datasets}. Esses dados serão divididos entre treino e teste, sendo que as amostras usadas para treinamento serão referentes a dados mais antigos em comparação com os dados usados para teste. Na Seção~\ref{sec:metricas}, são descritas métricas que serão usadas para avaliação de desempenho neste trabalho.


\subsection{Conjuntos de dados}\label{subsec:datasets}

Grande parte dos trabalhos que realizaram experimentos com dados de dinâmica da digitação não disponibilizaram os dados coletados~\cite{Roy2022systematic}. Esse fato dificulta a reprodutibilidade de estudos na área. Este projeto irá utilizar dados publicamente disponíveis. Alguns conjuntos de dados que podem ser usados são descritos a seguir:


\subsubsection{CMU}\label{subsubsec:cmu}

Este conjunto de dados~\footnote{\url{https://www.cs.cmu.edu/~keystroke/}} possui dados de 51 indivíduos que digitaram a senha ``.tie5Roanl'' em oito sessões de captura, com 50 amostras em cada sessão. No total, cada indivíduo digitou a senha 400 vezes. É importante destacar que, assim como em outros conjuntos de dados públicos e privados, houve um intervalo de tempo entre as sessões de coleta, para que as variações na digitação de cada usuário fossem consideradas no modelo. Cada pessoa participou de somente uma sessão por dia.

Os autores explicam que dentre os motivos da senha ser fixa para todos os usuários estão o viés que poderia surgir nos experimento se cada usuário pudesse escolher a própria senha e a necessidade de coletar amostras de impostores específicas para cada usuário, o que tornaria a coleta de dados ainda mais difícil.


\subsubsection{KeyRecs}\label{subsubsec:keyrecs}

O conjunto de dados KeyRecs~\footnote{\url{https://zenodo.org/records/7886743}} envolveu a captura de dinâmica da digitação de texto fixo e de texto livre. A princípio, o foco deste projeto será em texto fixo, portanto apenas essa parte do conjunto de dados deve ser utilizada. Para texto fixo, de acordo com a descrição do conjunto de dados, 99 indivíduos digitaram uma mesma senha em duas sessões, com 100 amostras em cada sessão, totalizando 200 amostras por indivíduo. Ao realizar o download da versão disponível, entretanto, observou-se que alguns usuários tem menos do que 200 amostras.


\subsection{Métricas}\label{sec:metricas}

Esta seção descreve algumas métricas usadas na literatura que serão usadas para avaliação dos resultados nos experimentos realizados neste projeto de pesquisa. Essas métricas são: FMR, FNMR e acurácia balanceada~\cite{Precise2014, Ferlini2021eargate}. Uma breve descrição dessas métricas é apresentadas a seguir:

\begin{itemize}
    \item FMR (\textit{False Match Rate}, Taxa de falsa correspondência): percentual de tentativas de impostores que foram aceitas como genuínas, definida como

          \begin{equation}\label{eqn:fmr}
              FMR = \frac{numero\:de\:tentativas\:de\:impostores\:aceitas}{total\:de\:tentativas\:de\:impostores}.
          \end{equation}

          Uma taxa relacionada é a FAR (\textit{False Acceptance Rate}), que tem significado similar, mas considera também taxa em que o sistema biométrico falha ao obter uma amostra biométrica. Essa taxa é conhecida como FTA (\textit{Failure to Acquire Rate}).


    \item FNMR (\textit{False Non-match Rate}, Taxa de falsa não-correspondência): percentual de tentativas genuínas que foram rejeitadas como impostoras pelo sistema, definida como

          \begin{equation}\label{eqn:fnmr}
              FNMR = \frac{numero\:de\:tentativas\:genuinas\:rejeitadas}{total\:de\:tentativas\:de\:usuarios\:genuinos}.
          \end{equation}

          Uma métrica relacionada é a FRR (\textit{False Rejection Rate}), que tem um significado similar, mas considera também a FTA.

    \item Acurácia balanceada: média do acerto para cada classe (genuíno e impostor). Essa métrica pode ser obtida a partir da o cálculo da (HTER -~\textit{Half Total Error}, Metade do erro total)

          \begin{equation}\label{eqn:hter}
              HTER = \frac{FNMR + FMR}{2},
          \end{equation}

          definida como a média entre FNMR e FMR~\cite{Roy2022systematic}. A partir da HTER, então é obtida a acurácia balanceada, definida como

          \begin{equation}\label{eqn:bacc}
              BAcc = 1 - HTER.
          \end{equation}

\end{itemize}

\subsection{Algorítmos e Hiperparâmetros}


