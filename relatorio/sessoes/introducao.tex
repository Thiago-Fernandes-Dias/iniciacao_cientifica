\section{Introdução}\label{sec:introduacao}

Nos últimos anos, a popularização de serviços oferecidos única e exclusivamente pela da Internet tem crescido de forma extraordinária. Muitas pessoas realizam pagamentos em sites de lojas diretamente de suas casas e gerenciam suas contas bancárias sem a necessidade de irem até os bancos~\cite{seguranca_de_dados}. Dessa forma, surge a necessidade sistemas de autenticação e identificação cada vez mais sofisticados.

Entretanto, na perspectiva de \citeonline{Ryu2023Design}, a maioria dos sistemas de autenticação comumente usados são baseados em senhas estáticas, ou uma combinação entre uma senha e um cartão inteligente ou \textit{tokens}.~\citeonline{Peacock2004} destacam que essa abordagem possui uma série de limitações. Por exemplo, senhas compostas por palavras comuns, sequências numéricas ou alfanuméricas simples que sejam passíveis de memorização são consideradas ``fracas'' no que se refere ao nível de segurança que elas oferecem, pois podem ser descobertas por ataques do tipo ``força bruta''.

~\citeonline{Peacock2004} ainda afirmam que é recomendado que os usuários criem senhas maiores e compostas por uma combinação de números, letras e caracteres especiais para serem menos vulneráveis a ataques. Em adição a isso, as senhas deveriam ser diferentes para cada conta nos sites em que acessam. Todavia, muitas vezes os usuários ainda optam pela utilização de uma mesma senha para todas as suas contas, o que gera um risco de segurança: se apenas um dos sistemas for invadido e tiver os seus dados roubados, o acesso aos dados do indivíduo em todos os outros serviços estará comprometido.

Segundo \citeonline{biometrics_a_tool_for_information_security}, muitas dessas limitações associadas ao uso de senhas podem ser contornadas pela incorporação de métodos de autenticação melhores. Nesse sentido, a biometria, como uma forma de estabelecer a identidade por meio de características físicas ou comportamentais dos indivíduos, surge como uma alternativa. Há diversas modalidades biométricas~\cite{Jain2016_50Years}.

Neste projeto, será dado o enfoque na vertente comportamental, em particular, na \textit{dinâmica de digitação}~\cite{Roy2022systematic}. Essa modalidade biométrica envolve analisar a forma com que um indivíduo digita em um teclado~\cite{monrose2000keystroke}. Os usuários são então reconhecidos com base no seu ritmo de digitação.

Conforme definido em~\cite{Jain2004_IntroToBiometrics}, sistemas biométricos são sistemas de reconhecimento de padrões que extraem características de dados biométricos e então comparam as características extraídas com uma referência biométrica em um banco de dados~\cite{Jain2004_IntroToBiometrics}. Essa comparação das características extraídas a partir dos dados biométricos com a referência biométrica no banco de dados frequentemente resulta em uma pontuação (\textit{score})~\cite{Jain2016_50Years}. Assumindo que seja uma pontuação indicando a similaridade, a classificação pode ser realizada aplicando um limiar de corte (\textit{threshold}). Se a pontuação for maior que o limiar, o dado biométrico é classificado como genuíno e, caso contrário, como sendo de um impostor.

Nesse contexto, o limiar de corte pode ser entendido com um hiperparâmetro. O trabalho de~\cite{hpo_foundations_algorithms_and_open_challenges} menciona o limiar de corte como um hiperparâmetro em algoritmos que retornam uma pontuação ou uma probabilidade. Outros algoritmos usados para reconhecimento de usuários pela dinâmica da digitação podem possuir outros hiperparâmetros para serem ajustados. Os valores dos hiperparâmetros tem impacto importante no desempenho preditivo de um sistema biométrico.

O objetivo deste projeto é comparar diferentes abordagens para ajuste de hiperparâmetros em algoritmos de classificação para dinâmica de digitação. Sobre esse aspecto, há algumas questões que podem ser investigadas. A primeira é sobre realizar o ajuste de forma global ou individualizada para cada usuário no sistema biométrico. Em biometria, o limiar de corte pode ser um valor global comum para todos os usuários ou um valor específico para cada usuário~\cite{Giot2011Unconstrained, Mhenni2019DoubleSerial}. Esse conceito pode ser estendido para outros hiperparâmetros, determinando os valores de forma comum a todos os usuários ou de forma individualizada. Outra questão que pode ser estudada neste contexto é sobre a técnica de ajuste de hiperparâmetros.

As demais seções do projeto estão organizadas da seguinte forma: na Seção~\ref{sec:fundamentacao_teorica}, são introduzidos conceitos sobre dinâmica da digitação e ajuste de hiperparâmetros; na Seção~\ref{sec:metodologia} são descritos os conjuntos de dados utilizados e suas particularidades e os experimentos realizados; na Seção~\ref{sec:resultados} são expostos os resultados e é feita uma discussão sobre eles e; na Seção~\ref{sec:conclusao} é feita a conclusão do projeto

