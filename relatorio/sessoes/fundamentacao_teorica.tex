\section{Fundamentação Teórica}\label{sec:fundamentacao_teorica}

Esta seção apresenta alguns conceitos importantes para este projeto de pesquisa envolvendo dinâmica da digitação e ajuste de hiperparâmetros.


\subsection{Dinâmica da digitação}\label{subsec:dinamica_de_digitacao}

A dinâmica da digitação é uma modalidade biométrica comportamental que diferencia os indivíduos com base em atributos característicos da digitação de textos, como o tempo em que o indivíduo permanece pressionando cada tecla, o intervalo de tempo entre cada ativação de tecla e padrões nos erros de digitação~\cite{biometric_personal_auth_using_keystroke_dynamics}.

Segundo~\citeonline{Peacock2004}, a principal vantagem da dinâmica da digitação em relação às demais modalidades biométricas comportamentais é a sua transparência. Por exemplo, se usada em conjunto com um formulário de autenticação comum, composto por um identificador de usuário (ou nome de usuário) e uma senha, as métricas de digitação podem ser obtidas das informações que o indivíduo necessariamente deverá inserir no sistema que ele deseja acessar. Além disso, em serviços baseados na Web, muitas vezes não é viável exigir formas de autenticação por biometria, pois os usuários podem não ter acesso aos dispositivos necessários, como câmeras e sensores de impressões digitais, ou equipamentos mais sofisticados.

Para fins de reconhecimento biométrico, os dados podem ser obtidos por meio dos padrões de digitação tanto de~\textit{textos fixos} definidos previamente, que os usuários serão requisitados a digitar para fins de identificação ou verificação, ou de~\textit{textos livres}, sem um tamanho fixo ou qualquer outra restrição. Muitos dos estudos desenvolvidos sobre autenticação pela dinâmica de digitação consideram um mecanismo baseado em texto fixo, geralmente o nome de usuário e a senha coletados previamente. Entretanto, diversos pesquisadores também aplicam algoritmos de aprendizado de máquina para desenvolver modelos capazes de autenticar os usuários de forma contínua por meio do texto digitado em um sistema durante o seu uso~\cite{continuous_auth_by_free_text_keystroke_based_on_cnn_and_rnn}.


\subsubsection{Coleta e modelagem de dados}\label{subsubsec:coleta_e_modelagem}

Considerando uma situação em que o texto a ser digitado é fixo, para criar modelos de Aprendizado de Máquina capazes de realizar a verificação de usuários a partir da forma com que eles digitam o texto pré-determinado é necessário um processo de treinamento a partir dos dados coletados durante a digitação do texto. Além disso, esses dados devem ser modelados corretamente para que sejam utilizados nos algoritmos de Aprendizado de Máquina.

Enquanto o usuário digita o texto são registrados os intervalos de tempo de ativação entre as teclas, sendo eles~\cite{Dias2023keyrecs}:

\begin{itemize}
    \item{\textit{Hold time}, ou~\textit{Dwell time}}: tempo em que o usuário permaneceu pressionando uma tecla após a ativação;
    \item{\textit{Down-Down}, ou~\textit{Flight time}}: intervalo de tempo entre a ativação de duas teclas consecutivas;
    \item{\textit{Down-Up}}: intervalo de tempo entre a ativação de uma tecla e a desativação da próxima tecla
    \item{\textit{Up-Down}}: intervalo de tempo entre a desativação de uma tecla e ativação da próxima tecla;
    \item{\textit{Up-Up}}: intervalo de tempo entre a desativação de duas teclas.
\end{itemize}

A cada vez que um participante digita o texto fixo, esses intervalos são registrados. Eles compõem um vetor, que serve como dado de entrada em algoritmos de Aprendizado de Máquina. 


\subsection{Hiperparâmetros}\label{subsec:ajuste_de_hiperparametros}

De acordo com~\citeonline{on_hyperparameter_optimization_of_machine_learning_algorithms}, desenvolver um modelo de aprendizado de máquina que seja efetivo na resolução de um determinado problema é uma tarefa complexa e demorada. Ela envolve a escolha do algoritmo apropriado e a obtenção de um modelo arquitetural ótimo por meio do ajuste de hiperparâmetros. O autor explica que há dois tipos de parâmetros em modelos de aprendizado de máquina: os~\textit{parâmetros} do próprio modelo, que serão inicializados e repetidamente atualizados durante o processo de treinamento, e os chamados~\textit{hiperparâmetros}, que devem ser escolhidos antes de o modelo ser treinado. Os autores mencionam os pesos dos neurônios em redes neurais como um exemplo de parâmetro de modelo e o valor de penalidade $C$ em uma~\textit{Support Vector Machine} (SVN), a taxa de aprendizado em redes neurais e o algoritmo utilizado para minimizar a função objetivo como hiperparâmetros.

O ajuste de hiperparâmetros é o processo de testar valores diferentes para os hiperparâmetros a fim de se obter o melhor ajuste para um modelo construído a partir de determinada base de dados. Na perspectiva de~\citeonline{automated_machine_learning}, o ajuste de hiperparâmetros é uma parte fundamental da construção de modelos de aprendizado de máquina efetivos, especialmente em redes neurais artificiais e modelos baseados em árvores de decisão, que possuem diversos hiperparâmetros.
~\citeonline{effect_of_the_sampling_of_dataset_in_hpo} explica que problemas de otimização de hiperparâmetros (\textit{hyperparameter optimization}, HPO) exigem um entendimento profundo da relação entre as combinações de hiperparâmetros e o modelo de aprendizado de máquina resultante do processo de treinamento. Ambos dependem do algoritmo utilizado e do tipo de cada hiperparâmetro, que pode ser contínuo, discreto ou categórico.

Segundo~\citeonline{on_hyperparameter_optimization_of_machine_learning_algorithms}, após a escolha do algoritmo de aprendizado de máquina e dos métodos que serão utilizados para avaliar o seu desempenho, é necessário listar os hiperparâmetros que deverão ser ajustados e, então, definir os conjuntos de valores possíveis para cada um de acordo com o seu tipo. Dependendo do problema, os hiperparâmetros podem possuir restrições, isto é, não poderão assumir qualquer valor dentre todos os valores possíveis, e essas restrições impostas a um hiperparâmetro podem estar condicionadas aos valores escolhidos para outro. Além disso, para cada configuração diferente, o modelo deverá ser treinado e testado novamente, para que o seu desempenho seja medido.

Alguns motivos para aplicar técnicas de ajuste de hiperparâmetros em aprendizado de máquina são evitar a necessidade de realizar o ajuste manualmente, o aprimoramento do desempenho dos algoritmos, assim como a melhora da reprodutibilidade e justiça dos estudos realizados~\citeonline{automated_machine_learning, hpo_foundations_algorithms_and_open_challenges}. A próxima seção discute algumas questões sobre o ajuste de hiperparâmetros no contexto de dinâmica da digitação, que será o foco deste projeto.


\subsection{Ajuste de hiperparâmetros no contexto de dinâmica da digitação}\label{trabalhos_relacionados}

Diversos trabalhos na área de dinâmica da digitação acabam não aplicando uma técnica de ajuste de hiperparâmetros em razão da métrica usada para reportar os resultados. Isso ocorre, por exemplo, ao reportar resultados em termos de EER (\textit{Equal Error Rate}). Ao ajustar o limiar de corte de um sistema biométrico, as taxas de falsa aceitação (impostores aceitos erroneamente) e de falsa rejeição (usuários genuínos rejeitados de forma indevida) podem mudar. De maneira geral, ao aumentar uma taxa, a outra diminui dependendo do ajuste do limiar de corte. O valor EER representa o ajuste em que as duas taxas são iguais~\cite{Roy2022systematic}. Para isso, os rótulos de teste (genuíno/impostor) podem ser usados para encontrar esse ajuste. Entretanto, em uma aplicação prática, o acesso aos rótulos dos dados pode não estar disponível.

Alguns trabalhos avaliaram o impacto dos hiperparâmetros. No trabalho elaborado por~\citeonline{one_vs_all_svm_implementation}, a técnica~\textit{Grid Search} foi aplicada para encontrar a melhor combinação de hiperparâmetros para uma implementação do algoritmo~\textit{Support Vector Machine} (SVM) usado para reconhecimento de usuários pela dinâmica da digitação. Os estudos realizados por~\citeonline{biometric_identification_based_on_keystroke_dynamics} avaliaram diferentes arquiteturas de redes neurais e a influência de hiperparâmetros como número de filtros convolucionais, tamanho do~\textit{kernel} de convolução, número de neurônios na camada recursiva e taxa de~\textit{drop out}.

Uma discussão sobre ajustar o limiar de corte de forma individual e de forma global foi realizada por~\citeonline{Giot2011Unconstrained}, assim como também avaliou a adaptação de modelos ao longo do tempo. De fato, em dinâmica da digitação, o ritmo de digitação pode mudar com o tempo. Outro trabalho que avaliou o ajuste de hiperparâmetros em dinâmica da digitação foi o de~\citeonline{keystroke_template_update_with_adapted_thresholds}. Nesse trabalho, foi considerado um cenário de sistemas biométricos adaptativos~\cite{Ryu2023Design}, em que a referência biométrica pode ser atualizada conforme os usuários realizam a autenticação. Os mesmos autores também discutiram essa adaptação do limitar de corte em~\cite{Mhenni2019DoubleSerial}.
