\section{Resultados e discussão dos resultados}\label{sec:resultados}

Nesta seção, serão apresentados os resultados dos experimentos. Nas tabelas a seguir, são apresentadas a $FNMR$, $FMR$ e $BAcc$ médias de cada modelo, em ambos os conjuntos de dados. O ajuste de hiperparâmetros, ou HPO (\textit{Hyperparameters Optimization}), foi feito de duas formas: por usuário e global. Na primeira, considerando um usuário, todas as configurações possíveis de hiperparâmetros foram testadas, e a que resultou em um modelo com maior $BAcc$ calculada com as amostras de validação foi selecionada. No HPO global, cada configuração de hiperparâmetros foi testada com todos os usuários, e a $BAcc$ média dos modelos foi calculada utilizando os dados de validação. A configuração que resultou em uma $BAcc$ média mais alta foi selecionada.

\begin{table}[htbp]
\centering
\caption{Médias das métricas dos experimentos utilizando o CMU}\label{tab:exp_metric_averages_cmu}
\begin{tabular}{|l|c|c|c|}
\hline
\textbf{Experimento} & \textbf{BACC} & \textbf{FNMR} & \textbf{FMR} \\
\hline
ST & 0.755 & 0.389 & 0.102 \\
\hline
ST com HPO global & 0.755 & 0.389 & 0.102 \\
\hline
ST com HPO por usuário & 0.778 & 0.345 & 0.099 \\
\hline
RF & 0.813 & 0.325 & 0.049 \\
\hline
RF com HPO global & 0.500 & 0.916 & 0.084 \\
\hline
RF com HPO por usuário & 0.809 & 0.332 & 0.049 \\
\hline
SVM & 0.534 & 0.874 & 0.058 \\
\hline
SVM com HPO global & 0.534 & 0.874 & 0.058 \\
\hline
SVM com HPO por usuário & 0.528 & 0.881 & 0.063 \\
\hline
\end{tabular}
\end{table}

\begin{table}[htbp]
\centering
\caption{Médias das métricas dos experimentos utilizando o Keyrecs}
\label{tab:exp_metric_averages_keyrecs}
\begin{tabular}{|l|c|c|c|}
\hline
\textbf{Experimento} & \textbf{BACC} & \textbf{FNMR} & \textbf{FMR} \\
\hline
ST & 0.749 & 0.147 & 0.355 \\
\hline
ST com HPO global & 0.765 & 0.230 & 0.240 \\
\hline
ST com HPO por usuário & 0.776 & 0.248 & 0.201 \\
\hline
RF & 0.820 & 0.284 & 0.076 \\
\hline
RF com HPO global & 0.500 & 0.926 & 0.074 \\
\hline
RF com HPO por usuário & 0.818 & 0.289 & 0.076 \\
\hline
SVM & 0.554 & 0.769 & 0.124 \\
\hline
SVM com HPO global & 0.555 & 0.768 & 0.122 \\
\hline
SVM com HPO por usuário & 0.557 & 0.760 & 0.125 \\
\hline
\end{tabular}
\end{table}

Na Tabela~\ref{tab:exp_metric_averages_cmu} é possível observar que a $BAcc$ média foi maior quando o ajuste por usuário foi utilizado, tanto para o Random Forest quanto para o algoritmo de~\citeonline{MagalhaesStatistical}. No SVM, no entanto, o uso do ajuste global causou uma ligeira diminuição na acurária, em comparação com o ajuste por usuário.

Outro ponto que vale a pena destacar é que o Random Forest deve um desempenho melhor no CMU do que no Keyrecs, e no primeiro conjunto de dados o ajuste por usuário causou uma diminuição significativa na $FNMR$. Com isso, a acurácia do modelo ficou acima de 90\%, como pode ser observado na Tabela~\ref{tab:exp_metric_averages_keyrecs}.

Como pode ser visto nas tabelas~\ref{tab:exp_metric_averages_cmu} e~\ref{tab:exp_metric_averages_keyrecs} o RF e o SVM tiveram resultados ruins, com uma FNMR alta e FMR baixa, o que significa que o modelo rejeitou a maioria das amostras, tanto do usuário genuíno quanto de impostores. Isso indica que o método de ajuste não foi adequado para estes algoritmos, e que um ajuste específico para cada usuário pode ser uma alternativa melhor se eles forem utilizados.

Para cada usuário foi comparada a acurácia dos modelos com cada abordagem de ajuste de hiperparâmetros. Nas Tabelas~\ref{tab:cmu_hp_comparison} e~\ref{tab:keyrecs_hp_comparison} é apresentada a quantidade de usuários que foram beneficiados pelo ajuste global ou individual, para cada um dos algoritmos utilizados, no CMU e no Keyrecs, respectivamente. 

\begin{table}[htbp]
\centering
\caption{Quantidade de usuários benefíciados pelo ajuste de hiperparâmetros no CMU}
\label{tab:cmu_hp_comparison}
\begin{tabular}{|l|c|c|c|}
\hline
\textbf{Algoritmo} & \textbf{Ajuste global} & \textbf{Ajuste por usuário} \\
\hline
ST & 14 (27.5\%) & 37 (72.5\%) \\
\hline
RF & 2 (3.9\%) & 49 (96.1\%) \\
\hline
SVM & 31 (60.8\%) & 20 (39.2\%) \\
\hline
\end{tabular}
\end{table}

\begin{table}[htbp]
\centering
\caption{Quantidade de usuários beneficiados pelo ajuste de hiperparâmetros no Keyrecs}
\label{tab:keyrecs_hp_comparison}
\begin{tabular}{|l|c|c|c|}
\hline
\textbf{Algoritmo} & \textbf{Ajuste global} & \textbf{Ajuste por usuário} \\
\hline
ST & 23 (23.2\%) & 76 (76.8\%) \\
\hline
RF & 0 (0.0\%) & 99 (100.0\%) \\
\hline
SVM & 52 (52.5\%) & 47 (47.5\%) \\
\hline
\end{tabular}
\end{table}

Em ambos os conjuntos de dados, somente quando o SVM foi utilizado houveram mais usuários beneficiados pelo ajuste global. Além disso, para todos os usuários do CMU e do Keyrecs o ajuste individual foi melhor quando o RF foi utilizado, e para o ST o ajuste individual foi melhor para a maioria dos usuários.

Nas figuras~\ref{fig:st_t_hist_cmu} e~\ref{fig:st_t_hist_keyrecs} é mostrada a quantidade usuários por valor aproximado de limiar de corte com uma linha vertical indicando o valor encontrado com o ajuste global, utilizando o ST no CMU e no Keyrecs, respectivamente. Como pode ser observado, houve uma quantidade maior de usuários com o limiar de corte próximo do valor global.

\begin{figure}[htbp]
    \centering
    \includegraphics[width=0.8\textwidth]{histograma_thresholds_medio_cmu.png}
    \caption{Quantidade de usuários por valor de limiar de corte no CMU, utilizando o CMU}\label{fig:st_t_hist_cmu}
\end{figure}

\begin{figure}[htbp]
    \centering
    \includegraphics[width=0.8\textwidth]{histograma_thresholds_medio_keyrecs.png}
    \caption{Quantidade de usuários por valor de limiar de corte no CMU, utilizando o CMU}\label{fig:st_t_hist_keyrecs}
\end{figure}
