\section{Resultados e discussão dos resultados}\label{sec:resultados}

Nesta sessão, serão apresentados os resultados dos experimentos. Nas tabelas abaixo, são apresentadas a $FNMR$, $FMR$ e $BAcc$ médias de cada modelo, em ambos os conjuntos de dados

\begin{table}[htbp]
\centering
\caption{Médias das métricas dos experimentos utilizando o CMU}
\label{tab:exp_metric_averages_cmu}
\begin{tabular}{|l|c|c|c|}
\hline
\textbf{Experimento} & \textbf{BACC} & \textbf{FRR} & \textbf{FAR} \\
\hline
Magalhães & 0.755 & 0.389 & 0.102 \\
\hline
Magalhães com HPO global & 0.755 & 0.389 & 0.102 \\
\hline
Magalhães com HPO por usuário & 0.771 & 0.365 & 0.093 \\
\hline
Random Forest & 0.813 & 0.325 & 0.049 \\
\hline
Random Forest com HPO global & 0.500 & 0.916 & 0.084 \\
\hline
Random Forest com HPO por usuário & 0.945 & 0.055 & 0.055 \\
\hline
SVM & 0.534 & 0.874 & 0.058 \\
\hline
SVM com HPO global & 0.534 & 0.874 & 0.058 \\
\hline
SVM com HPO por usuário & 0.528 & 0.881 & 0.063 \\
\hline
\end{tabular}
\end{table}

\begin{table}[htbp]
\centering
\caption{Médias das métricas dos experimentos utilizando o Keyrecs}
\label{tab:exp_metric_averages_keyrecs}
\begin{tabular}{|l|c|c|c|}
\hline
\textbf{Experimento} & \textbf{BACC} & \textbf{FRR} & \textbf{FAR} \\
\hline
Magalhães & 0.749 & 0.147 & 0.355 \\
\hline
Magalhães com HPO global & 0.750 & 0.313 & 0.187 \\
\hline
Magalhães com HPO por usuário & 0.774 & 0.249 & 0.204 \\
\hline
Random Forest & 0.820 & 0.284 & 0.076 \\
\hline
Random Forest com HPO global & 0.500 & 0.926 & 0.074 \\
\hline
Random Forest com HPO por usuário & 0.818 & 0.289 & 0.076 \\
\hline
SVM & 0.554 & 0.769 & 0.124 \\
\hline
SVM com HPO global & 0.555 & 0.768 & 0.122 \\
\hline
SVM com HPO por usuário & 0.557 & 0.760 & 0.125 \\
\hline
\end{tabular}
\end{table}

Na primeira tabela é possível observar que a $BACC$ média foi maior quando o ajuste por usuário foi utilizado, tanto para o Random Forest quanto para o algoritmo de~\citeonline{MagalhaesStatistical}. No SVM, no entanto, o uso do ajuste global causou uma ligeira diminuição na acurária, em comparação com o ajuste por usuário.

Outro ponto que vale a pena destacar é que o Random Forest deve um desempenho melhor no CMU do que no Keyrecs, e no primeiro conjunto de dados o ajuste por usuário causou uma diminuição significativa na $FRR$. Com isso, a acurácia do modelo ficou acima de 90\%.

Nas tabelas abaixo é apresentada a quantidade de usuários que foram beneficiados pelo ajuste global ou individual, para cada um dos algoritmos utilizados. Uma para cada conjunto de dados.

\begin{table}[htbp]
\centering
\caption{Quantidade de usuários benefícios pelo ajuste de hiperparâmetros no CMU}
\label{tab:cmu_hp_comparison_extended}
\begin{tabular}{|l|c|c|c|}
\hline
\textbf{Algoritmo} & \textbf{Ajuste Global} & \textbf{Ajuste por Usuário} \\
\hline
ST & 14 (27.5\%) & 37 (72.5\%) \\
\hline
RF & 0 (0.0\%) & 51 (100.0\%) \\
\hline
SVM & 31 (60.8\%) & 20 (39.2\%) \\
\hline
\end{tabular}
\end{table}

\begin{table}[htbp]
\centering
\caption{Quantidade de usuários benefícios pelo ajuste de hiperparâmetros no Keyrecs}
\label{tab:keyrecs_hp_comparison_extended}
\begin{tabular}{|l|c|c|c|}
\hline
\textbf{Algoritmo} & \textbf{Ajuste global} & \textbf{Ajuste por usuário} \\
\hline
ST & 23 (23.2\%) & 76 (76.8\%) \\
\hline
RF & 0 (0.0\%) & 99 (100.0\%) \\
\hline
SVM & 52 (52.5\%) & 47 (47.5\%) \\
\hline
\end{tabular}
\end{table}

