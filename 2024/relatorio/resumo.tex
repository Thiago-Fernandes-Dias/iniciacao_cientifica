\section*{Resumo}

O uso de biometria para autenticação de usuários, em oposição a métodos tradicionais baseados em senhas estáticas, têm sido atrativo em razão da maior segurança que um sistema biométrico pode proporcionar. Os usuários podem ser reconhecidos com base em características físicas ou comportamentais ao invés de precisarem se lembrar de uma senha ou a ter em mãos um cartão inteligente ou~\textit{token}. Dentre as diversas modalidades biométricas existentes, há a~\textit{dinâmica de digitação}, que reconhece as pessoas com base em seu ritmo de digitação. Diversos algoritmos de classificação podem ser usados nesse contexto. Esses algoritmos possuem hiperparâmetros, que precisam ser ajustados adequadamente. O ajuste de hiperparâmetros pode ser realizado para cada usuário individualmente ou pode ser global, isso é, os hiperparâmetros assumirão o mesmo valor para todos os usuários. O objetivo deste projeto é explorar diferentes abordagens de ajuste de hiperparâmetros em algoritmos para classificação para dinâmica de digitação.