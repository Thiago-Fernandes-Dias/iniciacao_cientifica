\section{Introdução}\label{sec:introduacao}

Por volta dos anos de 1970, conforme explica~\citeonline{handbook_of_biometrics}, engenheiros da IBM sugeriram que usuários de sistemas computacionais poderiam ser autenticados com base em suas características físicas ou comportamentais, únicas para cada pessoa. Com isso, seria possível limitar o acesso a informação somente a pessoas autorizadas de forma segura, sem a necessidade de a pessoa possuir algum tipo de cartão ou~\textit{token} de autenticação ou ter de memorizar uma senha, que poderia ser roubada por diversos meios.

Em~\cite{handbook_of_biometrics} são descritos detalhadamente uma gama de modalidades biométricas físicas, como impressão digital, padrão de veias nas mãos e íris, e comportamentais, como a caligrafia ou assinatura feita à mão. Cada uma possui a suas vantagens e desvantagens, além de diversas aplicações nos sistemas de autenticação atuais. Neste projeto, será dado o enfoque na vertente comportamental. Em particular, na~\textit{dinâmica de digitação}~\cite{Roy2022systematic}, que é uma modalidade biométrica que envolve analisar a forma com que um indivíduo digita em um teclado~\cite{monrose2000keystroke}. Os usuários são, então, reconhecidos com base no seu ritmo de digitação.

Conforme definido em~\cite{Jain2004_IntroToBiometrics}, sistemas biométricos são sistemas de reconhecimento de padrões que extraem características de dados biométricos e então comparam as características extraídas com uma referência biométrica em um banco de dados. Essa comparação das características extraídas com a referência biométrica no banco de dados frequentemente resulta em uma pontuação (\textit{score})~\cite{Jain2016_50Years}. Assumindo que seja uma pontuação indicando a similaridade, a classificação pode ser realizada aplicando um limiar de corte (\textit{threshold}). Se a pontuação for maior que o limiar, o dado biométrico é classificado como genuíno e, caso contrário, como sendo de um impostor.

Nesse contexto, o limiar de corte pode ser entendido como um hiperparâmetro. O trabalho de~\citeonline{hpo_foundations_algorithms_and_open_challenges} menciona o limiar de corte como um hiperparâmetro em algoritmos que retornam uma pontuação ou uma probabilidade. Outros algoritmos usados para reconhecimento de usuários pela dinâmica da digitação podem possuir outros hiperparâmetros para serem ajustados. Os valores dos hiperparâmetros tem um grande impacto no desempenho preditivo de um sistema biométrico.

\textbf{O objetivo deste projeto é comparar diferentes abordagens para ajuste de hiperparâmetros em algoritmos de classificação para dinâmica de digitação}. Sobre esse aspecto, há algumas questões que podem ser investigadas. A primeira é sobre realizar o ajuste de forma global ou individualizada para cada usuário no sistema biométrico. Em biometria, o limiar de corte e a configuração de hiperparâmetros pode ser global, isso é, comum para todos os usuários, ou uma configuração específica para cada usuário~\cite{Giot2011Unconstrained, Mhenni2019DoubleSerial}.

As demais seções do projeto estão organizadas da seguinte forma: na Seção~\ref{sec:fundamentacao_teorica}, são introduzidos conceitos sobre dinâmica da digitação e ajuste de hiperparâmetros; na Seção~\ref{sec:metodologia} são descritos os conjuntos de dados utilizados e suas particularidades e os experimentos realizados; na Seção~\ref{sec:resultados}, são expostos os resultados e é feita uma discussão sobre eles e; na Seção~\ref{sec:conclusao}, é feita a conclusão do projeto.

