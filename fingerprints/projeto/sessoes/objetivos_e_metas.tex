\section{Breve descrição dos objetivos e metas}
\label{sec:objetivos}

Este projeto tem o objetivo de \textbf{comparar diferentes abordagens para ajuste de hiperparâmetros de algoritmos de classificação em dinâmica da digitação}. Para isso, será avaliado o ajuste de hiperparâmetros de forma individual e global. Além disso, diferentes técnicas de ajuste de hiperparâmetros podem ser investigadas nesse contexto.

A princípio, o foco do projeto será na dinâmica de digitação de \textit{texto fixo}, em que todos os indivíduos digitam a mesma expressão. O desempenho será avaliado por métricas como FMR, FNMR e acurácia balanceada, descritas na Seção \ref{sec:metricas}.

Os objetivos específicos do projeto são:

\begin{itemize}
    \item Selecionar algoritmos de classificação usados em dinâmica da digitação;
    \item Definir abordagens de ajuste de hiperparâmetros que possam ser aplicadas em dinâmica da digitação (por exemplo: ajuste individual, ajuste global);
    \item Realizar experimentos comparando as diferentes abordagens de ajuste;
    \item Avaliar desempenho obtido pelos algoritmos com cada abordagem.
\end{itemize}
